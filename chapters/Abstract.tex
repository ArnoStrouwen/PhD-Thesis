\chapter*{Abstract}
Living systems are rife with variability. Well-chosen experimental designs are necessary to deal with this variability when modeling such systems. Models for living systems often rely on knowledge of physical, chemical and biological laws, such as mass balances, transport phenomena and reaction kinetics, and are often described by a system of non-linear differential equations. So, the structure of a model can often be determined from first principles. However, the model will generally also rely on parameters whose numerical values cannot be determined from physical laws. These parameters must then be inferred from experimental data before the model can be put to use. A well-chosen experiment can greatly improve the estimation of the model parameters. But there exist several challenges for constructing such informative experiments for dynamic systems. One challenges is the dependence of the optimal experiment on the true model parameters, making it difficult to perform robust experiments that work well regardless of the specific model parameter values. Another challenge is the correlation of the observations due to the presence of process noise. The central research topic of this thesis revolves around solving these challenges by developing robust experimental design methodology for noisy dynamic systems. My novel methodology is developed to improve postharvest applications, and is mainly applied to the estimation of respiration and fermentation parameters of pear fruit. 
\\
\\
After an introduction in the first chapter, the second chapter of this PhD thesis deals with the estimation of respiration parameters of pear fruit inside a jar, modeled by Michaelis-Menten kinetics. Air flowing into the jar has to be controlled so that the parameters of the Michaelis-Menten model can be estimated as precisely as possible. The quality of this parameter estimation, and thus of the experimental design, is quantified using the determinant of the Fisher information matrix, which is inversely related to the area of the confidence ellipse of the two Michaelis-Menten model parameters. The air flowing into the jar must thus be optimized so that the determinant of the Fisher information matrix is as large as possible. One major challenge here is the dependence of this Fisher information matrix on the true, but unknown, parameters of the system. The most commonly used method in the literature to deal with this issue is locally optimal design, where a single initial guess is used for the true parameter. One main conclusion from my work is that this optimal experimental design technique outperforms several commonly used heuristic experimental techniques.
\\
\\
The third chapter builds on the first but also takes fermentation of the pear fruit into account. The locally optimal design method only takes into account a single initial guess, and may not perform well if this guess deviates substantially from the true parameter values. Instead of a single initial guess, an entire distribution could be used to quantify the prior knowledge and uncertainty on the unknown parameters. Because of the use of a prior distribution, this method is called Bayesian experimental design. Most current techniques in the literature only allow for parametric prior distributions, such as normal distributions. The prior information about respiration and fermentation, coming from a previously gathered dataset, could not be summarized by any parametric distribution. For this reason, I developed a novel experimental design technique based on a Markov-chain Monte-Carlo (MCMC) analysis of this previously gathered data. This method is thus able to approximate arbitrary distributions. I found that this flexible experimental design technique is more robust than the commonly used locally optimal design method.
\\
\\
The fourth chapter focuses on robust and adaptive experimental design techniques for dynamic systems with process noise. Current experimental design techniques for dynamic systems generally only incorporate measurement noise, but biological systems also often involve process noise. Calculating the Fisher information matrix for such systems requires estimating the uncertain dynamic states, using Bayesian filtering techniques. For linear dynamical systems, the optimal filter is the Kalman filter. But deriving the Fisher information matrix for dynamic systems under process noise and then applying the methodology from the previous chapters is not sufficient to construct informative experiments. This is due to the difficulty in precisely predicting such systems far into the future, which causes those future measurements to contribute little to the Fisher information matrix. Adaptive experimental designs are able to deal with this issue. Adaptive designs use the already gathered data to re-optimize and thus adapt the remainder of the experiment. The already gathered measurements can help by reducing the uncertainty on the model parameters, which the optimal design depends upon. Adaptivity is thus always a good design strategy, even when no process noise is present. But for systems with process noise, adaptivity is even more important, since the already gathered measurements help increase the prediction accuracy of future measurements, thus increasing the informativity of these measurements. I found that taking into account process noise in the experimental design greatly improves the quality of the experiment.