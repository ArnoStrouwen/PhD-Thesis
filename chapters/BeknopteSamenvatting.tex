\chapter*{Beknopte Samenvatting}
Levende systemen worden gekenmerkt door een grote maat aan variabiliteit. Goed gekozen experimentele
ontwerpen zijn noodzakelijk om goed om te gaan met deze variabiliteit. Modellen van levende systemen steunen vaak op kennis van fysieke, chemische of biologische wetten, zoals massabalansen, transportverschijnselen en reactiekinetiek, en worden vaak beschreven door een systeem van niet-lineaire differentiaalvergelijkingen. Dus, de structuur van het model kan vaak bepaald worden door eerste principes. Echter, het model zal in het algemeen ook afhangen van parameters waarvan de numerieke waarden niet kunnen vastgesteld worden door fysieke wetten. Deze parameters moeten dan afgeleid worden uit experimentele data voordat het model toegepast kan worden. Een goed gekozen experiment kan de schatting van deze parameters sterk verbeteren. Er bestaan echter verschillende uitdagingen bij het opstellen van zo een informatief experiment voor dynamische systemen. Een eerste uitdaging is de afhankelijkheid van het optimale experiment van de waarden van de modelparameters, wat het moeilijk maakt om robuuste experimenten op te zetten die goed werken voor om het even welke modelparameterwaarden. Een andere uitdaging is de correlatie tussen de observaties door de aanwezigheid van procesruis. Het centrale onderzoeksthema van deze thesis draait rond het aanpakken van deze uitdagingen door het ontwerpen van een robuuste methodologie voor experimenteel ontwerp voor dynamische systemen met ruis. Mijn nieuwe methodologie is ontworpen om naoogsttoepassingen te verbeteren, en is voornamelijk gericht op de schatting van respiratie- en fermentatieparameters van peren.
\\
\\
Na een inleiding in het eerste hoofdstuk, behandelt het tweede hoofdstuk van deze doctoraatsthesis het schatten van respiratieparameters van peer in een pot, gemodelleerd met behulp van Michaelis-Menten kinetiek. Lucht die de pot instroomt kan aangestuurd worden zodat de parameters van het Michaelis-Menten model zo precies mogelijk kunnen
geschat worden. De kwaliteit van deze parameterschatting, en dus ook het experimenteel ontwerp, wordt gekwantificeerd door middel van de determinant van de Fisher informatiematrix, die omgekeerd gerelateerd is aan de betrouwbaarheidsellips van de twee Michaelis-Menten modelparameters. Lucht die de pot instroomt moet geoptimaliseerd worden zodat de determinant van de Fisher informatiematrix zo groot mogelijk is. Een grote uitdaging is het feit dat deze Fisher informatiematrix afhangt van de (onbekende) modelparameters.  De meest gebruikte techniek in de literatuur voor het omgaan met deze kwestie is lokaal optimaal ontwerp, waarbij een enkele initiële gok wordt gebruikt voor de echte parameters. De belangrijkste conclusie van mijn werk is dat dit optimaal experimenteel ontwerp verscheidene vaak gebruikte heuristische methodes van experimenteel ontwerp overtreft.
\\
\\
Het tweede hoofdstuk bouwt verder op het eerste maar houdt ook rekening met fermentatie van peer fruit. Het lokaal optimaal ontwerp houdt alleen rekening met een enkele initiële gok, en presteert doorgaans niet goed als deze gok substantieel afwijkt van de echte parameterwaarden. In de plaats van een enkele initiële gok kan een ganse kansverdeling gebruikt worden om de a priori kennis en onzekerheid omtrent de modelparameters te kwantificeren. Omwille van het gebruik van een a priori verdeling wordt deze methode Bayesiaans experimenteel ontwerp genoemd. De meeste huidige technieken in de literatuur staan enkel parametrische a priori verdelingen toe. De a priori informatie over respiratie en fermentatie, waarover ik beschikte, kon niet samengevat worden door een parametrische kansverdeling. Om deze reden ontwierp ik een nieuwe techniek gebaseerd op een Markov chain Monte-Carlo analyse van deze voorheen verzamelde data. Deze methode kan arbitraire kansverdelingen benaderen. Ik ontdekte dat deze flexibele techniek robuuster is dan de vaak gebruikte techniek van lokaal optimaal ontwerp.
\\
\\
Het derde hoofdstuk richt zich op robuuste en adaptieve technieken voor experimenteel otwerp voor dynamische systemen met procesruis. De huidige technieken voor dynamische systemen kunnen enkel omgaan met meetruis, maar procesruis komt ook vaak voor in biologische systemen. Het berekenen van de Fisher informatiematrix voor dergelijke systemen vereist het schatten van de onzekere dynamische toestanden met behulp van Bayesiaanse filtertechnieken. Voor lineaire dynamische systemen is de optimale filter de Kalman filter. Echter, het afleiden van de Fisher informatiematrix voor dynamische systemen onder procesruis en vervolgens de methodologie uit de voorgaande hoofdstukken toepassen volstaan niet om informatieve experimenten op te stellen. Dit is te wijten aan de moeilijkheid om dergelijke systemen ver in de toekomst nauwkeurig te voorspellen, waardoor metingen ver in de toekomst weinig bijdragen aan de Fisher informatiematrix. Adaptieve experimentele ontwerpen zijn in staat om dit probleem te remediëren. Adaptieve ontwerpen gebruiken de reeds verzamelde gegevens om de rest van het experiment opnieuw te optimaliseren en dus aan te passen aan de vrijgekomen informatie. De reeds verzamelde metingen helpen de onzekerheid te verminderen omtrent de modelparameters, waarvan het optimale ontwerp afhankelijk is. Adaptiviteit is dus altijd een goede ontwerpstrategie, ook als er geen procesruis aanwezig is. Maar voor systemen met procesruis is adaptiviteit nog belangrijker dan voor systemen zonder procesruis, omdat de reeds verzamelde metingen de nauwkeurigheid van de voorspellingen van toekomstige metingen helpen te vergroten, waardoor de informativiteit van deze metingen toeneemt. In mijn derde hoofdstuk demonstreerde ik dat het beschouwen van procesruis in het experimentele ontwerp de kwaliteit van het experiment aanzienlijk verbetert.
