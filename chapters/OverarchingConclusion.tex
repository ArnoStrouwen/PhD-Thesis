\chapter{Conclusion and Outlook}
\label{conclusion}
\section{Overarching Conclusion}
The general goal of this thesis is to show the usefulness of dynamic experiments for postharvest applications. The case studies in Chapters \ref{paper1} and \ref{paper2} show that dynamic experiments indeed lead to a much more precise estimation of respiration and fermentation parameters of pear fruit with reduced experimental effort. This thesis also contains a methodological contribution in Chapter \ref{paper3} for experimental design in the presence of both process noise and measurement noise. 

To reach this general goal, multiple challenges had to be overcome:
\begin{itemize}
	\item \textbf{Challenge 1}: The optimal experiment depends on the true values of the model parameters which the experiment aims to learn about. Some form of prior information about these parameters is thus always required to begin optimizing the experiment. In \textbf{Chapter \ref{paper1}}, we solved this issue by taking parameter estimates from the literature and used them to construct locally optimal designs. Using this methodology, experiments to estimate Michaelis-Menten respiration parameters for pear fruit were constructed. These experiments perform better than heuristic techniques used in post-harvest experimentation.
	\item \textbf{Challenge 2}: Robust experimental designs perform well over a wide range of possible parameter values. Bayesian experimental design is a popular technique to achieve robustness, where the prior information is quantified using a probability distribution. The design is then optimized to perform well on average, where the information content is averaged over this prior distribution. Available Bayesian experimental design techniques from the literature only allow for parametric distributions to quantify this prior information, and these techniques were found to be insufficient to adequately describe the prior information we had available for the respiration and fermentation of pear fruit. In \textbf{Chapter \ref{paper2}}, an experimental design methodology was developed that can deal with arbitrary prior distributions obtained by using a Markov-Chain Monte-Carlo integration technique. This methodology was then found to perform better than the traditional locally optimal design method used in \textbf{Chapter \ref{paper1}}. This methodology was then used to construct robust experiments to estimate the respiration and fermentation of pear fruit.
	\item \textbf{Challenge 3}: Little experimental design literature exists on how to deal with dynamic systems with both process and measurement noise. In \textbf{Chapter \ref{paper3}}, the Fisher information matrix (FIM) is derived for linear dynamical systems with both kinds of noise. The key discovery here is that experimental design for dynamic systems with hidden dynamic states requires tracking these hidden states using an appropriate Bayesian filter, which for linear systems equals the Kalman filter.  
	\item \textbf{Challenge 4}: Adaptive experimental design techniques are needed when the prior information about model parameters is poor. These adaptive techniques re-optimize the remainder of the experimental design after every measurement. However, the optimization problem that occurs in the traditional experimental design technique, which is based on the expected Fisher information matrix, keeps growing in complexity at every time-step. In \textbf{Chapter \ref{paper3}}, this issue is solved by developing a novel experimental design criterion based on a combination of the observed and expected FIM. The observed FIM is computationally less expensive as it only utilizes the actually observed measurements, thus making it well suited for quantifying the information from the already gathered measurements. The expected FIM requires an expectation over all possible observations, making it well suited to quantify the information of future observations. A combination of these two types of Fisher information matrices results in an experimental design methodology that requires the same amount of computation at every time-step, which is a necessity for adaptive experimental design.
\end{itemize} 