\chapter*{Preface}
During my bachelor studies as a bio-science engineer at KU Leuven, I followed a course on optimal experimental design by Professor Peter Goos. In this course, it became clear to me that statistical modeling is essential to properly describe living systems, due to the inherently high variability of such systems. It also became clear that carefully planned experiments are necessary to cope with this variability. The knowledge I gained from this course made me quite popular among my fellow bio-science engineering students, since the informative cost- and time-efficient experiments I planned for them saved them multiple days (and nights!) of lab work.
\\
\\
Thus, after completing my master studies, I knew immediately which direction I wanted to give to my further career. I e-mailed Professor Goos about a PhD in his research group. To my surprise, Professor Goos remembered me as the only student in his academic career who did not fail his exam, but nevertheless came to the ask feedback on their exam. Apparently, this left a good impression and I could soon start working on the thesis you are now reading. The first project Professor Goos suggested to me involved choice experiments with mixture constraints. I was already quite interested in this topic and was ready to agree to it, but, mostly out of curiosity, I asked whether there existed experimental design research about dynamic systems, since my master thesis focused heavily on dynamic systems and I had also considered doing a PhD in control theory. Here, I must say that the stars aligned as Professor Goos was just getting involved in a project about the metabolism of pear during hypoxia (together with Professors Bart Nicolaï and Wouter Saeys), which required dynamic experiments. I was truly lucky to be able to combine my two research interests, control theory and experimental design, and I cannot imagine a research topic that would have interested me more. And thus it came to be that my thesis would be about optimal experimental design for dynamic systems, applied to the storage of pear fruit, with Professor Goos advising me on experimental design and Professor Nicolaï advising me on postharvest modeling.
Peter, I must thank you for teaching me to write a sound scientific manuscript. I am sitting besides a stack of all the drafts of our papers and they easily tower above me. Bart, your practical knowledge of postharvest models always rescued me when I got stuck in my PhD project.
\\
\\
Only a couple of months into my PhD, I was thrown to the lions of the Research Foundation - Flanders (FWO) strategic basic research panel. Thankfully, I made it out alive, and I am very grateful that the FWO believed in me and decided to fund my research for four years. 
\\
\\
I want to thank the members of my supervisory comity, Professors Kristel Bernaerts and Wouter Saeys, for the constructive feedback throughout my PhD project. In particular, the PhD thesis of Professor Bernaerts and her many papers were a great starting point for my own research.
\\
\\
I also want to express my gratitude towards Professor Julio Banga and Doctor Philippe Nimmegeers for agreeing to be part of my jury.
\\
\\
Professor Banga is further thanked for agreeing to a research visit of mine to his research group in Vigo. I learned a lot about sensitivity analysis of dynamic systems and global optimization during this visit.
\\
\\
I also want to thank the first master thesis student I supervised, Karel Van Brantegem. Karel, you wrote an excellent master thesis and I am glad that you decided to join our research group. As you know by now, I have more research ideas than I could conceivable finish myself. So, it was really helpful to be able to collaborate with you to tackle some of them.
\\
\\
The last person I must thank on the academic side of things is somebody I have only communicated with in text, but was nevertheless instrumental in bringing this thesis to a successful end. This is Doctor Christopher Rackauckas, without whose scientific machine learning project, the text of this thesis would probably be twice as long, and the computer source code, upon which it is based, would be four times longer.
\\
\\
In the year 1993 my grandparents retired, but instead of enjoying their well-deserved free time, they got a new job. A baby was born in their family and the parents had very demanding careers. Bomma and Bompa without your love and care I would never have achieved half as much.
\\
\\
Arno,
\\
Hasselt 2021

