\section{Outlook}
In this thesis, I developed novel experimental design methodology for dynamic systems. I believe that more interesting and important research can still be done in this area, including research on experimental design methodology for more complicated dynamic models, novel ways to quantify uncertainty and more efficient numerical techniques. In this section, I discuss some possible perspectives on future research.
\subsection{Model Selection}
It is commonly assumed in the experimental design literature that the entire model structure is known. However, it is often more realistic that:
\begin{itemize}
\item Competing model structures exist to explain a certain phenomenon.
\item Only part of the model structure is known, and there are missing components in the model.
\end{itemize}
Appropriate experimental design techniques should be developed that can deal with both model parameter and model structure uncertainty.
\subsubsection{Model Discrimination}
Some experimental design techniques for model discrimination in dynamic systems exist \parencite{chen}, mostly based on the Hunter and Reiner criterion \parencite{hunter}. A weakness of this approach is that it is not robust, since it does not account for parameter uncertainty. Instead, the method only compares models at point estimates of the parameters. If there is much parameter uncertainty, the Hunter and Reiner method will be overly optimistic in its ability to discriminate models. Instead of this method, a Bayesian optimal design method, similar to the one used in Chapter \ref{paper2}, could be developed, where an information criterion, averaged out over a joint prior distribution for the model structure and the model parameters, is optimized. The challenge here is to generalize the Markov-Chain Monte-Carlo integration technique to deal with both the continuous model parameters and the discrete model structures. 
\subsubsection{Missing Model Components}
Another form of model uncertainty occurs when only part of the model structure is known. One method to deal with missing model components is to replace them with a black box model, such as a Gaussian process or neural network. Replacing parts of a differential equation with a surrogate model is currently a hot topic within machine learning \parencite{rackauckas3,rackauckas4}. Optimal experimental design techniques have not been used by machine learning communities, as they often deal with large observational datasets. However, when combining the black box modeling approach from machine learning and mechanistic models for biological systems, in scenarios where acquiring experimental data is cumbersome and time-consuming, there is a need for carefully crafted experiments. Instead of quantifying the uncertainty of a finite dimensional parameter space, here the challenge lies in quantifying the uncertainty of an infinite dimensional function space. To achieve this, methodological concepts from optimal design of computer experiments can be borrowed, where the Fisher information and entropy based criteria have already been combined with Gaussian processes \parencite{santner}. Of course, computer simulations are often deterministic, while biology is rife with variability.
\\
\\
Throughout this thesis we assumed that Michaelis-Menten reaction kinetics perfectly summarize the respiration and fermentation behavior of pear fruit. However, in reality this is just a summary of a much larger biological pathway, and diffusion also plays a role, leading to a high dimensional partial differential equation model. Instead of working directly with this more complicated model, the misspecification on the lumped model could be taken into account through a Gaussian process \parencite{kennedy}, but experimental design for such misspecified systems has not yet been considered in literature.
\subsection{Process Noise}
In chapter \ref{paper3}, I developed an experimental design method to deal with process noise for discrete-time linear dynamical systems. For such systems, the Fisher information matrix has a tractable form, involving the Kalman filter to estimate the hidden state. The next step here is replacing the linear dynamics with non-linear dynamics. The Kalman filter then no longer suffices to estimate the hidden state. A more complicated filter, such as an extended Kalman, unscented or particle filter must then be used, greatly increasing the computational cost. If we then also consider continuous time models, we arrive at the unexplored research topic of experimental design for stochastic differential equation models.
\subsection{Kullback-Leibler Divergence}
In this thesis, I used the Fisher information matrix (FIM) to quantify the information content of an experiment. The Kullback-Leibler divergence (KL-div) is another method to quantify the information content. Unlike the FIM, the KL-div does not involve normal approximations, and might be a better criterion for small sample sizes. The major drawback is that the KL-div is numerically much more intensive to calculate.  This is generally done using a double loop Monte-Carlo integration method \parencite{ryan}. Recently, advances have been made in lowering the computational burden of the KL-div based approach by considering variational Bayesian techniques \parencite{foster}, where the inner Monte-Carlo loop is replaced by optimizing a variational distribution. This technique then allows for jointly optimizing the design and variational parameters \parencite{foster2}.
\subsection{Koopman Expectation}
In Chapter \ref{paper2} of this thesis, we calculated the expectation required for Bayesian experimental design by pushing forward the uncertain extended state distribution in time, and then calculating some quantity of interest at each time point. \textcite{gerlach} recently proposed a new method to calculate expectations for dynamic systems that instead pulls back this quantity of interest in time, using the Koopman operator which relies on the change of variable formula. This idea is analogous to the back propagation (adjoint sensitivity analysis) of gradients for optimization. Calculating the expectation using the Koopman operator is thus generally a good choice when a low number of expectations have to be calculated, such as the expectation of a scalar function as opposed to the expectation of a vector function. In Bayesian experimental design, such an expectation of a scalar function of the FIM is calculated. This means calculating the expectation using the Koopman operator should be favorable from a computational cost perspective. 
\subsection{Fully Risk Averse Experimental Design}
In this thesis, we took an expected value approach to deal with uncertainty, in the sense that uncertainty on the model parameters was quantified using a probability distribution, and experiments were optimized for the average of an information criterion over this distribution. Experimental design techniques that instead assume a worst case scenario for the model parameter uncertainty also exist, such as the techniques developed by \textcite{bauer}. Here, the experimental design is optimized for the minimum of an information criterion over the set of possible model parameters. This method, however, only uses the worst case scenario to deal with parameter uncertainty, but there is, of course, also uncertainty in the measurements, and other model disturbances such as process noise. If these errors are bounded, we can also assume a worst case scenario here. There exist parameter estimation techniques that produce the entire set of model parameters that are consistent with the measurements, while eliminating those that are not possible \parencite{jaulin}. However, these set based techniques have only recently been used for experimental design \parencite{jauberthie}. It would in particular be interesting to combine these methods with reachability analysis for dynamic systems \parencite{juliareach}, which are techniques to numerically solve differential equations, which return a tube in which the true solution is guaranteed to exist. Fully risk averse experimental design techniques would particularly be of interest for designing experiments that guarantee the true model parameters lie within a set of a chosen size.  
\subsection{Goal Driven Experimental Design}
Most experimental design literature for dynamic systems focuses on a precise estimation of all model parameters, putting equal importance on each parameter. Often, the reason for estimating the model parameters is to use the resulting model in the control and optimization of the bioprocess. To achieve an optimal control or to find the optimal input settings for a bioprocess, not all model parameters are equally important. Instead, it is typically a non-linear function of a subset of the model parameters that matters. More effort should therefore be spent on precisely estimating the relevant functions of the influential parameters. The challenge here is transforming the variability of the parameters into a variability of optimal operation levels. Experiments then must be constructed to minimize this latter uncertainty.
\\
\\
This topic of precisely estimating a non-linear function has only been briefly explored in the dynamic experimental design literature, more specifically by \textcite{houska}. These authors started by using the Fisher information matrix to quantify the uncertainty about the model parameters and then linearize the effect that the model parameters have on the optimal conditions and transform the parameter uncertainty into an uncertainty on the optimal conditions, using the delta method. However, they did not use this technique to find an optimal point in the design region, which would be a particularly interesting application of this technique. \textcite{li} suggested a method to achieve this, but only for generalized linear models. Their method splits the Kullback-Leibler divergence into two parts: (i) the Kullback-Leibler divergence between the observations and the unknown parameters, and (ii) an information loss term between the unknown parameters and the optimal conditions quantified by the data processing inequality theorem \parencite{cover}. However, this technique has not yet been used in combination with dynamic systems.
